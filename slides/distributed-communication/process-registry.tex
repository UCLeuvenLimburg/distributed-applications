\section{Process Registration}

\frame{\tableofcontents[currentsection]}

\begin{frame}
    \frametitle{What is a process registry?}
    \begin{itemize}
        \item Process with a key-value datastructure 
        \item Maps "identifiers" with PID's
    \end{itemize}
\end{frame}

\begin{frame}
    \frametitle{What is a process registry?}
    Your \texttt{Assignment.ProcessManager}!!! % TODO Update this if necessary
    \vfill
    But how do we do this distributed?
\end{frame}

\begin{frame}
    \frametitle{Libraries}
    \begin{itemize}
        \item :global : Erlang global process registry
        \item ostinelli/syn: global process registry \& group manager
        \item bitwalker/swarm: global process registry \& so much more
    \end{itemize}
    
    \vfill

    (We'll use :global for now.)
\end{frame}

\begin{frame}
    \frametitle{Goal: Global Processes}
    On node A: \\
    \texttt{\{:ok, pid\} = GenServer.start\_link(.. , .. , [name: ProcessA])} 

    \vfill

    On node B: \\
    \texttt{Process.whereis(ProcessA) \# will return nil} 
\end{frame}

\begin{frame}
    \frametitle{Goal: Global Processes}
    On node A: \\
    \texttt{\{:ok, pid\} = GenServer.start\_link(.. , .. , [name: ProcessA])} \\
    \texttt{:global.register\_name(ProcessA, p)}

    \vfill

    On node B: \\
    \texttt{Process.whereis(ProcessA) \# will still return nil} \\ 
    \texttt{:global.whereis\_name(ProcessA) \# will return pid} \\ 
\end{frame}

\begin{frame}
    \frametitle{Goal: Global Processes}
    Note: you can register a process as a tuple. \\ 
    (\texttt{\{Node.self(), \_\_MODULE\_\_\}} is common usage.)

    \vfill

    Short demo\\
    \tiny{(demo code is in the exercises folder)}
\end{frame}
