\section{Linked processes}

\frame{\tableofcontents[currentsection]}

\begin{frame}
    \frametitle{2 processes (\texttt{\&spawn\_link/1})}
    \begin{center}
        \begin{tikzpicture}
            \coordinate (process A position) at (0,0);
            \coordinate (process B position) at (2,0);

            \only<1-3>{
                \node[process] (process A) at (process A position) {A};
            }
            \only<4>{
                \node[dead process] (process A) at (process A position) {A};
            }

            \only<1-2>{
                \node[process] (process B) at (process B position) {B};
            }
            \only<3-4>{
                \node[dead process] (process B) at (process B position) {B};
            }

            \draw[-latex] (process A) -- (process B);
        \end{tikzpicture}
    \end{center}
    \vskip5mm
    \begin{overprint}
        \onslide<1>
        \begin{center}
            Process A initiates link
        \end{center}
        \vskip4mm
        \code[language=elixir,font=\small,width=.95\linewidth]{linking.exs}

        \onslide<2>
        \begin{center}
            Processes are now linked \textbf{bidirectionally}
        \end{center}

        \onslide<3>
        \begin{center}
            Process B dies
        \end{center}

        \onslide<4>
        Process A receives a ''death message``, and if it is not a system process, no custom behaviour can be implemented when this message is received.
        \vfill
        \textit{Note: Even if it is a system process, the end result should also be death.
        This is often used when you want to do certain actions, such as cleaning up your process.}
    \end{overprint}
\end{frame}
